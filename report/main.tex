\documentclass{article}
\nocite{*}

% Language setting
% Replace `english' with e.g. `spanish' to change the document language
\usepackage[english]{babel}

% Set page size and margins
% Replace `letterpaper' with`a4paper' for UK/EU standard size
\usepackage[letterpaper,top=2cm,bottom=2cm,left=3cm,right=3cm,marginparwidth=1.75cm]{geometry}

% Useful packages
\usepackage{amsmath}
\usepackage{graphicx}
\usepackage[colorlinks=true, allcolors=blue]{hyperref}

\title{Selfie Tool for the Visually Impaired}
\author{Ethan Arnold, Alexander Cohentest}

\begin{document}
\maketitle


\section{Development Process}
This project was pair-programmed, with us simultaneously using the computer.


For the project, we decided to use OpenCV for facial recognition.
We imported 'cv2' and opened the camera as a video feed.
The facial detection returns a rectangle with a coordinate and dimensions.
We determine the position of the rectangle by finding the center of it, which is the position plus half the dimension respective to the axes.
The Quadrants of the page are determined by a range of values.
The video frame is a two-dimensional grid of pixels, so one can take the half of a dimension and use comparisons (greater than, less than) to determine if it is on the top/bottom or left/right side of the screen.
Compounding this idea with both axes, you can then determine if the rectangle is in one of four quadrants.



\section{Usage Statistics}
\section{Similar Applications}
\section{Potential Improvement}



\bibliographystyle{alpha}
\bibliography{sample}

\end{document}